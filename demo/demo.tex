\documentclass[10pt,aspectratio=169]{beamer}

\usetheme{metropolis}
\usepackage{appendixnumberbeamer}

% use different fonts
\usepackage{xcharter-otf}
\usepackage{FiraSans}
\usepackage{inconsolata}
\setsansfont{Fira Sans}
\setmainfont{Fira Sans}
\setmonofont{inconsolata}

\usepackage{booktabs}
\usepackage[scale=2]{ccicons}

\usepackage{pgfplots}
\usepgfplotslibrary{dateplot}

\usepackage{xspace}
\newcommand{\themename}{\textbf{\textsc{metropolis}}\xspace}

\title{Metropolis}
\subtitle{A modern beamer theme}
\date{\today}
\author{Matthias Vogelgesang}
\institute{Center for modern beamer themes}
% \titlegraphic{\hfill\includegraphics[height=1.5cm]{logo.pdf}}

\renewcommand<>{\emph}[1]{%
  {\usebeamercolor[fg]{emph}\only#2{\itshape}#1}%
}

\usepackage{algpseudocode}

\begin{document}
\metroset{block=fill} % to fill block environments
\nocite{beamer} % cite the Beamer package

\maketitle

\begin{frame}{Table of contents}
  \setbeamertemplate{section in toc}[sections numbered]
  \tableofcontents[hideallsubsections]
\end{frame}

\section{Simple Proofs}

\begin{frame}{Irrationality of $\mathbf{\sqrt{2}}$}
	\begin{theorem}
	The square root of two is irrational.
	\end{theorem}
 	The following proof uses the \emph{fundamental theorem of arithmetic}.
	\begin{proof}
	For the sake of contradiction, assume that $\sqrt{2}$ is rational. Hence, there are integers $m,n \neq 0$ such that $\sqrt{2} = \frac{m}{n}$ or rather $\sqrt{2} \cdot n = m$.
        Squaring both sides yields $2 \cdot n^2 = m^2$. Clearly a contradiction.
	\end{proof}
\end{frame}

\begin{frame}{Infinitude of Primes}
	\begin{theorem}
	There are infinitely many primes.
	\end{theorem}
	\begin{lemma}
	The value of Riemann zeta function $\zeta(2)$ is trancendental, namely,
	\begin{align*}\zeta(2) = \prod\limits_{p \in \mathbb{P}} \frac{1}{1 - \frac{1}{p^2}} = \frac{\pi^2}{6}.\end{align*}
	\end{lemma}
	\begin{proof}
	For the sake of contradiction, assume that there are only finitely many primes. Hence, $\zeta(2)$ is rational. Clearly a contradiction.
	\end{proof}
\end{frame}

\section{Graph Examples}

\begin{frame}{Properties of the Complete Graph}
Let $G=K_7$ be the complete graph with seven vertices. $G$ has several properties.
\begin{example}
\begin{itemize}[<+- | alert@+>]
	\item There is no vertex cover with only five vertices.
	\item $G$ is not planar\alert<4>{\only<4>{ (only the $K_1$, $K_2$, $K_3$, and $K_4$ are)}.}
	\item $G$ has an Eulerian cycle with length 21.
\end{itemize}
\end{example}
\end{frame}

\section{Complexity Theory}

\begin{frame}{Turing (Cook) Reductions}
Recall that \textsc{sat} and \textsc{taut} are \textbf{NP}-complete and \textbf{coNP}-complete, respectively.
\begin{theorem}
\textbf{NP} and \textbf{coNP} are indistinguishable with respect to Cook reductions.
\end{theorem}
\begin{proof}
We show that $\textsc{sat} \leq_C \textsc{taut}$ and then $\textsc{taut} \leq_C \textsc{sat}$. Let $\varphi$ be a formula.
\begin{enumerate}
	\item Note that $\varphi$ is satisfiable \textit{iff} $\neg \varphi$ is not a tautology.
	
	\item Note that $\varphi$ is a tautology \textit{iff} $\varphi$ is satifiable and $\neg\varphi$ is not.
\end{enumerate}
Hence, the respective oracles can be used as follows:
\begin{columns}[T,onlytextwidth]
    \column{0.5\textwidth}
\begin{algorithmic}[1]
        \Procedure{sat}{$\varphi$}
            \State \textbf{return} $\neg \textsc{taut}(\neg \varphi)$
        \EndProcedure
    \end{algorithmic}
    \column{0.5\textwidth}
	\begin{algorithmic}[1]
        \Procedure{taut}{$\varphi$}
            \State \textbf{return} $\textsc{sat}(\varphi) \land \neg \textsc{sat}(\neg \varphi)$
        \EndProcedure
    \end{algorithmic}
    \end{columns}
\end{proof}
\end{frame}

\section{Conclusion}

\begin{frame}{Summary}

  Get the source of this theme and the demo presentation from

  \begin{center}\url{github.com/matze/mtheme}\end{center}

  The theme \emph{itself} is licensed under a
  \href{http://creativecommons.org/licenses/by-sa/4.0/}{Creative Commons
  Attribution-ShareAlike 4.0 International License}.

  \begin{center}\ccbysa\end{center}

\end{frame}

\begin{frame}[standout]
  Questions?
\end{frame}

\appendix

\begin{frame}[fragile]{Backup slides}
  Sometimes, it is useful to add slides at the end of your presentation to
  refer to during audience questions.

  The best way to do this is to include the \verb|appendixnumberbeamer|
  package in your preamble and call \verb|\appendix| before your backup slides.

  \themename will automatically turn off slide numbering and progress bars for
  slides in the appendix.
\end{frame}

\begin{frame}[allowframebreaks]{References}

  \bibliography{demo}
  \bibliographystyle{abbrv}

\end{frame}

\end{document}
