\documentclass[10pt,aspectratio=169,english]{beamer}
\usepackage{mtheme-req}


\title{Metropolis}
\subtitle{A modern beamer theme}
\date{\today}
\author{Matthias Vogelgesang}
\institute{Center for modern beamer themes}
% \titlegraphic{\hfill\includegraphics[height=1.5cm]{logo.pdf}}


% Used for the graph example
\usetikzlibrary{graphs.standard}
\definecolor{niceRed}{HTML}{d20000}
\definecolor{niceOrange}{HTML}{f59a23}
\definecolor{niceGreen}{HTML}{7ec636}
\definecolor{niceBlue}{HTML}{007aff}

\tikzset{graphs/layout1/.style = {
    nodes={fill=nicerBlue!20, draw=nicerBlue, line width=1.2pt, outer sep=.55pt,minimum size=1.3em, circle}, clockwise, empty nodes, n=#1,
    edges={line width=1.5pt}
    },
    every picture/.style={line width=1.5pt}
}

\begin{document}
\metroset{block=fill} % fill block environments
\nocite{beamer} % cite the Beamer package

\maketitle

\begin{frame}{Table of Contents}
  \setbeamertemplate{section in toc}[sections numbered] % enumerate ToC 
  \tableofcontents[hideallsubsections]
\end{frame}


%%%%%%%%%%%%%%%%%%%%%%%%%%%%%%%
%%%%%%%% Simple Proofs %%%%%%%%
%%%%%%%%%%%%%%%%%%%%%%%%%%%%%%%


\section{Simple Proofs}

\begin{frame}{Irrationality of $\mathbf{\sqrt{2}}$}
	\begin{theorem}
	The square root of two is irrational.
	\end{theorem}
 	The following proof uses the \emph{fundamental theorem of arithmetic}.
	\begin{proof}[Proof\nopunct]
	For the sake of contradiction, assume that $\sqrt{2}$ is rational. Hence, there are integers $m,n \neq 0$ such that $\sqrt{2} = \frac{m}{n}$ or rather $\sqrt{2} \cdot n = m$.
        Squaring both sides yields $2 \cdot n^2 = m^2$. Clearly a contradiction.
	\end{proof}
\end{frame}


%%%%%%%%%%%%%%%%%%%%%%%%%%%%%%%
%%%%%% Complexity Theory %%%%%%
%%%%%%%%%%%%%%%%%%%%%%%%%%%%%%%


\section{Complexity Theory}

\begin{frame}{Turing (Cook) Reductions}
Recall that \textsc{sat} and \textsc{taut} are \textbf{NP}-complete and \textbf{coNP}-complete, respectively.
\begin{theorem}
\textbf{NP} and \textbf{coNP} are indistinguishable with respect to Cook reductions.
\end{theorem}
\begin{proof}[Proof\nopunct]
We show that $\textsc{sat} \leq_C \textsc{taut}$ and then $\textsc{taut} \leq_C \textsc{sat}$. Let $\varphi$ be a formula. Note that
\begin{enumerate}
	\item $\varphi$ is satisfiable \textit{iff} $\neg \varphi$ is not a tautology.
	
	\item $\varphi$ is a tautology \textit{iff} $\varphi$ is satisfiable and $\neg\varphi$ is not.
\end{enumerate}
Hence, the respective oracles can be used as follows:
\begin{columns}[T,onlytextwidth]
    \column{0.5\textwidth}
\begin{algorithmic}[1]
        \Procedure{sat}{$\varphi$}
            \State \textbf{return} $\neg \textsc{taut}(\neg \varphi)$
        \EndProcedure
    \end{algorithmic}
    \column{0.5\textwidth}
	\begin{algorithmic}[1]
        \Procedure{taut}{$\varphi$}
            \State \textbf{return} $\textsc{sat}(\varphi) \land \neg \textsc{sat}(\neg \varphi)$
        \EndProcedure {\qedhere\text{\,\,}}
    \end{algorithmic}
    \end{columns}
\end{proof}
\end{frame}


%%%%%%%%%%%%%%%%%%%%%%%%%%%%%%%
%%%%%%%% Graph Example %%%%%%%%
%%%%%%%%%%%%%%%%%%%%%%%%%%%%%%%


\section{Graph Properties}

\begin{frame}{Petersen Graph $P_{7,2}$\only<2->{: }\only<2>{Properties}\only<3>{Vertex Coloring}\only<4>{Hamiltonicity}}
Consider the generalized Petersen graph $P_{7,2}$\only<1>{:}\only<2->{.}

\only<1>{
\begin{figure}[h!]

% Generalized Petersen graph P_{7,2} (n=7, k=2)
\begin{tikzpicture}
  \graph[layout1=7, radius=6.5em] {subgraph C_n [name=A]};
  \graph[layout1=7, radius=3.5em] {subgraph I_n [name=B]};
  
  \def\n{7} \def\k{2} 
  \foreach \i in {1,...,\n}{ \draw (A \i) to (B \i); }
  \foreach \i [evaluate={\j=int(1+mod(int(\i+\k-1),\n))}] in {1,...,\n} { \draw (B \i) to (B \j); }
\end{tikzpicture}
\caption{The Petersen graph $P_{7,2}$}
\end{figure}
}
\vspace*{.3cm}
\only<2->{

\begin{columns}[T]
\begin{column}{.40\textwidth}

	It
	\begin{itemize}
		\item<3-> is 3-colorable,
		\item<4-> is Hamiltonian.
	\end{itemize}

\end{column}%
\hfill%
\begin{column}{.40\textwidth}

\begin{figure}[h!]

\only<2>{
\begin{tikzpicture}
  \graph[layout1=7, radius=6.5em] {subgraph C_n [name=A]};
  \graph[layout1=7, radius=3.5em] {subgraph I_n [name=B]};
  
  \def\n{7} \def\k{2} 
  \foreach \i in {1,...,\n}{ \draw (A \i) to (B \i); }
  \foreach \i [evaluate={\j=int(1+mod(int(\i+\k-1),\n))}] in {1,...,\n} { \draw (B \i) to (B \j); }
\end{tikzpicture}
}

% 3-colorable
\only<3>{
\tikzstyle{col1} = [fill=niceRed!20, draw=niceRed]
\tikzstyle{col2} = [fill=niceBlue!20, draw=niceBlue]
\tikzstyle{col3} = [fill=niceOrange!20, draw=niceOrange]
\tikzset{every node/.append style={line width=1.5pt, outer sep=.73pt, minimum size=1.3em, circle}}

\begin{tikzpicture}[x=1em,y=1em]
  \graph[layout1=7, radius=6.5em] {subgraph C_n [name=A]};
  \graph[layout1=7, radius=3.5em] {subgraph I_n [name=B]};
  
	% outer radius=6.5em
	\foreach \x in {1,6} \node[col1] (A \x) at ({6.5*sin(360/7*(\x-1))},{6.5*cos(360/7*(\x-1))}) {};
	\foreach \x in {2,4} \node[col2] (A \x) at ({6.5*sin(360/7*(\x-1))},{6.5*cos(360/7*(\x-1))}) {};
	\foreach \x in {3,5,7} \node[col3] (A \x) at ({6.5*sin(360/7*(\x-1))},{6.5*cos(360/7*(\x-1))}) {};
	% inner radius=3.5em
	\foreach \x in {2,5} \node[col1] (B \x) at ({3.5*sin(360/7*(\x-1))},{3.5*cos(360/7*(\x-1))}) {};
	\foreach \x in {3,6,7} \node[col2] (B \x) at ({3.5*sin(360/7*(\x-1))},{3.5*cos(360/7*(\x-1))}) {};
	\foreach \x in {1,4} \node[col3] (B \x) at ({3.5*sin(360/7*(\x-1))},{3.5*cos(360/7*(\x-1))}) {};

  
  \def\n{7} \def\k{2} 
  \foreach \i in {1,...,\n}{\pgfmathsetmacro\res{int(max(1, mod(\i+1,\n+1)))}
  	\draw (A \i) to (A \res);
  	\draw (A \i) to (B \i);
  }
  \foreach \i [evaluate={\j=int(1+mod(int(\i+\k-1),\n))}] in {1,...,\n} { \draw (B \i) to (B \j); }
\end{tikzpicture}
}

% Hamiltonian
\only<4>{
\tikzstyle{hc} = [niceBlue]
\begin{tikzpicture}
  \graph[layout1=7, radius=6.5em] {subgraph C_n [name=A]};
  \graph[layout1=7, radius=3.5em] {subgraph I_n [name=B]};
  
  \def\n{7} \def\k{2} 
  \foreach \i in {1,...,\n}{ \draw (A \i) to (B \i); }
  \foreach \i [evaluate={\j=int(1+mod(int(\i+\k-1),\n))}] in {1,...,\n} { \draw (B \i) to (B \j); }
  \draw[hc] (A 1) to (A 2);
  \draw[hc] (A 2) to (B 2);
  \draw[hc] (B 2) to (B 4);
  \draw[hc] (B 4) to (A 4);
  \draw[hc] (A 4) to (A 3);
  \draw[hc] (A 3) to (B 3);
  \draw[hc] (B 3) to (B 1);
  \draw[hc] (B 1) to (B 6);
  \draw[hc] (B 6) to (A 6);
  \draw[hc] (A 6) to (A 5);
  \draw[hc] (A 5) to (B 5);
  \draw[hc] (B 5) to (B 7);
  \draw[hc] (B 7) to (A 7);
  \draw[hc] (A 7) to (A 1);
\end{tikzpicture}
}

\end{figure}%
%
\end{column}%
\end{columns}

}

\end{frame}


%%%%%%%%%%%%%%%%%%%%%%%%%%%%%%%
%%%%%%%%%% Conclusion %%%%%%%%%
%%%%%%%%%%%%%%%%%%%%%%%%%%%%%%%


\section{Conclusion}

\begin{frame}{\translate{Summary}}

  Get the source of this theme and the demo presentation from

  \begin{center}\url{github.com/m3g33/blue-mtheme}\end{center}

  The theme \emph{itself} is licensed under a
  \href{http://creativecommons.org/licenses/by-sa/4.0/}{Creative Commons
  Attribution-ShareAlike 4.0 International License}.

  \begin{center}\ccbysa\end{center}

\end{frame}

\begin{frame}[standout]
  \translate{Questions}?
\end{frame}

\appendix

\begin{frame}[fragile]{Backup slides}
  Sometimes, it is useful to add slides at the end of your presentation to
  refer to during audience questions.

  The best way to do this is to include the \verb|appendixnumberbeamer|
  package in your preamble and call \verb|\appendix| before your backup slides.

  \themename will automatically turn off slide numbering and progress bars for
  slides in the appendix.
\end{frame}

\begin{frame}[allowframebreaks]{\translate{References}}

  \bibliography{demo}
  \bibliographystyle{abbrv}

\end{frame}

\end{document}
